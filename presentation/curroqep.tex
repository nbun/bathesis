\documentclass{beamer}
\usepackage{ulem}
\usepackage{proof}
\usepackage{amsmath}
\usepackage{amssymb}
\usepackage{minted}
\usemintedstyle[haskell]{trac}
\newminted{haskell}{fontsize=\small}
\newminted{coq}{fontsize=\small}
\setmintedinline{style = bw}
\newcommand{\coqinline}[1]{\mintinline{coq}{#1}}

\mode<presentation>
{
  \usetheme{Antibes}
  \usecolortheme{lily}
}


\usepackage[english]{babel}
% or whatever

\usepackage[utf8]{inputenc}
% or whatever

\usepackage{times}
\usepackage[T1]{fontenc}
\setbeamersize{text margin left=1.5em,text margin right=1.5em}

\title{Formalizing inference systems in Coq by means of type systems for Curry}

%\subtitle{Bachelor's Thesis} % (optional)

\author{Niels Bunkenburg}

\institute{ 
	Programming Languages and Compiler Construction \par
	Department of Computer Science \par
	Christian-Albrechts-University of Kiel}
% - Use the \inst command only if there are several affiliations.
% - Keep it simple, no one is interested in your street address.

\date[Short Occasion] % (optional)
{29.09.2016}

\subject{Formalizing inference systems in Coq by means of type systems for Curry}
% This is only inserted into the PDF information catalog. Can be left
% out. 



% If you have a file called "university-logo-filename.xxx", where xxx
% is a graphic format that can be processed by latex or pdflatex,
% resp., then you can add a logo as follows:

 %\pgfdeclareimage[height=0.5cm]{university-logo}{../Bilder/cau_siegel.png}
 %\logo{\pgfuseimage{university-logo}}



% Delete this, if you do not want the table of contents to pop up at
% the beginning of each subsection:
\AtBeginSubsection[]
{
  \begin{frame}<beamer>{Outline}
    \tableofcontents[currentsection,currentsubsection]
  \end{frame}
}


% If you wish to uncover everything in a step-wise fashion, uncomment
% the following command: 

%\beamerdefaultoverlayspecification{<+->}


\begin{document}

\begin{frame}
  \titlepage
\end{frame}

\begin{frame}{Motivation}

\end{frame}
\begin{frame}{Outline}
  \tableofcontents
  % You might wish to add the option [pausesections]
\end{frame}

\section{Introduction}
\subsection{Programming Languages}
\begin{frame}[fragile]{Coq - Data Types and Functions}
\begin{itemize}
\item Inductive definitions
\begin{coqcode}
Inductive list {X : Type} : Type :=
  | nil  : list X
  | cons : X -> list X -> list X.
\end{coqcode}
\vfill
\pause
\item (Recursive) functions
\begin{coqcode}
Fixpoint app {X: Type} (l1 l2: list X) : (list X) :=
  match l1 with
  | nil => l2
  | cons h t => cons h (app t l2)
  end.
\end{coqcode}
\end{itemize}
\end{frame}
\begin{frame}[fragile]{Coq - Propositions}
\begin{itemize}
\item Equations
\begin{coqcode}
1 + 1 = 2.
forall (X : Type) (l : list X), l ++ [] = l.
\end{coqcode}
\vfill
\item Inductively Defined Propositions
\begin{coqcode}
Inductive inInd : nat -> list nat -> Prop :=
 | head: forall n l, inInd n (n :: l)
 | tail: forall n l e, inInd n l -> inInd n (e :: l).
\end{coqcode}
\end{itemize}
\end{frame}
\begin{frame}[fragile]{Curry}
\begin{itemize}
\item Syntax similar to Haskell
\item Nondeterminism
\begin{haskellcode}
(?)   :: a -> a -> a
x ? _ = x
_ ? y = y
\end{haskellcode}
\item Free variables
\begin{coqcode}
1 + 1 == x where x free
\end{coqcode}
\end{itemize}
\end{frame}
\subsection{Theory}
\begin{frame}{Typing}
\end{frame}
\begin{frame}{Context}
\end{frame}
\begin{frame}[fragile]{Inference rules}
\begin{center}
	\begin{minipage}{.45 \linewidth}
		\infer[\text{\texttt{In\_H}}]{\coqinline{In n (n :: l)}}{\phantom{premise}}
	\end{minipage}
	\hspace{.1 \linewidth}
	\begin{minipage}{.45 \linewidth}
		\infer[\text{\texttt{In\_T}}]{\coqinline{In n (e :: l)}}{\coqinline{In n l}}
	\end{minipage}
\end{center}
\end{frame}
\section{CuMin}
\subsection{Modeling}
\begin{frame}{Syntax - Backus–Naur Form}
\begin{align*}
P &::= D;P \:|\:D \\
D &::= f :: \kappa \tau; f \overline{x_{n}} = e\\
\kappa &::= \forall^{\epsilon} \alpha.\kappa \:|\: \forall^{*}\alpha.\kappa \:|\: \epsilon \\
\tau &::= \alpha \:|\: \text{Bool} \:|\: \text{Nat} \:|\: [\tau] \:|\: (\tau,\tau ') \:|\: \tau \rightarrow \tau ' \\
e &::= x \:|\: f_{\overline{\tau_{m}}} \:|\: e_{1}\: e_{2} \:|\: \text{let}\: x = e_{1} \:\text{in}\: e_{2} \:|\: n \:|\: e_{1} + e_{2} \:|\: e_{1} \circeq e_{2}\\
&\:|\: (e_{1},e_{2}) \:|\: \text{case}\: e \:\text{of}\: \langle (x,y) \rightarrow e_{1}\rangle\\
&\:|\: \text{True} \:|\: \text{False} \:|\: \text{case}\: e \:\text{of}\: \langle \text{True} \rightarrow e_{1};\:\text{False} \rightarrow e_{2}\rangle\\
&\:|\: \text{Nil}_{\tau} \:|\: \text{Cons}(e_{1}, e_{2}) \:|\: \text{case}\: e \:\text{of}\: \langle \text{Nil} \rightarrow e_{1};\:\text{Cons}(x,y) \rightarrow e_{2}\rangle\\
&\:|\: \text{failure}_{\tau} \:|\: \text{anything}_{\tau}
\end{align*}
\end{frame}
\begin{frame}
\begin{minipage}{.45 \linewidth}
	\begin{align*}
	&\text{fst}::\forall^{*}\alpha.\forall^{*}\beta.(\alpha, \beta) \rightarrow\alpha\\
	&\text{fst}\: p = \text{case}\: p\: \text{of}\: \langle(u,v) \rightarrow u\rangle
	\end{align*}
\end{minipage}
\begin{minipage}{.45 \linewidth}
	\begin{align*}
	&\text{one} :: \text{Nat}\\
	&\text{one} = \text{fst}_{Nat, Bool} ~ (1, \text{True})
	\end{align*}
\end{minipage}
\end{frame}
\begin{frame}[fragile]{Syntax - Coq}
\begin{coqcode}
Inductive quantifier : Type :=
  | for_all : id -> tag -> quantifier.
	
Inductive ty : Type :=
  | TVar  : id -> ty
  | TBool : ty
  | TNat  : ty
  | TList : ty -> ty
  | TPair : ty -> ty -> ty
  | TFun  : ty -> ty -> ty.

Definition program := list func_decl.
Inductive func_decl : Type :=
  | FDecl : id -> list quantifier ->
	ty -> list id -> tm -> func_decl.
\end{coqcode}
\end{frame}
\begin{frame}{Context}
\end{frame}
\subsection{Typing}
\begin{frame}{Typing rules}
\end{frame}
\begin{frame}{Examples}
\end{frame}
\section{FlatCurry}
\subsection{Differences to CuMin}
\begin{frame}{Syntax}
\end{frame}
\subsection{Typing}
\section{Conclusion}
\begin{frame}{Summary}
\end{frame}
\begin{frame}{Future Work}
\end{frame}

\end{document}


